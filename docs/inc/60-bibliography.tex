\section*{СПИСОК ИСПОЛЬЗОВАННЫХ ИСТОЧНИКОВ}
\addcontentsline{toc}{section}{СПИСОК ИСПОЛЬЗОВАННЫХ ИСТОЧНИКОВ}

\begingroup
\renewcommand{\section}[2]{}
\begin{thebibliography}{}
	\bibitem{IDC} IDC [Электронный ресурс]. -- Режим доступа: https://www.idc.com/ (Дата обращения: 12.11.2021).
	
	\bibitem{IDC2020} Data Creation and Replication Will Grow at a Faster Rate than Installed Storage Capacity, According to the IDC Global DataSphere and StorageSphere Forecasts [Электронный ресурс]. -- Режим доступа: https://www.idc.com/getdoc.jsp?containerId=prUS47560321 (Дата обращения: 12.11.2021).
	
	\bibitem{IDC2018} Data Age 2025: the datasphere and data-readiness from edge to core [Электронный ресурс]. -- Режим доступа: https://www.i-scoop.eu/big-data-action-value-context/data-age-2025-datasphere/ (Дата обращения: 12.11.2021).
	
	\bibitem{IDC2025} How the pandemic impacted data creation and storage [Электронный ресурс]. -- Режим доступа: https://www.i-scoop.eu/big-data-action-value-context/data-storage-creation/ (Дата обращения: 12.11.2021).
	
	\bibitem{ICE} Еникеев, Р. Д. Двигатели внутреннего сгорания. Основные термины и русско-английские соответствия : учеб. пособие для вузов / Р. Д. Еникеев, Б. П. Рудой -- М. : Машиностроение, 2004. -- 383 с. -- Библиогр.: с. 378-382. -- ISBN 5-217-03267-7.
	
	\bibitem{isystem} Гаврилова, Т.А. Базы знаний интеллектуальных систем [Текст] / Т.А. Гаврилова, В.Ф. Хорошевский. -- СПб: Питер, 2000. -- 384 с. -- ISBN 5-272-00071-4.
	
	\bibitem{autonomy_cite} Autonomy. Autonomy Technology Whitepaper. -- 1998. [Электронный ресурс]. -- Режим доступа: http://www.autonomy.com (Дата обращения: 26.11.2021).
	
	\bibitem{autonomy} Олейник, Е. HP Autonomy IDOL: анализ совсем неструктурированных данных / Storage News --  2012. -- № 3 (51). -- С. 28-31 / [Электронный ресурс]. -- Режим доступа: http://www.storagenews.ru/51/HP\_Autonomy\_51.pdf (Дата обращения: 20.11.2021).
	
	\bibitem{marri} Громов, Ю.Ю. Интеллектуальные информационные системы и технологии:
	учебное пособие [Текст] /  Ю.Ю. Громов, О.Г. Иванова, В.В. Алексеев и др. -- Тамбов: Изд-во ФГБОУ ВПО «ТГТУ», 2013. -- 244 с. -- ISBN 978-5-8265-1178-7.
	
	\bibitem{philosophy} Макеева, Л. Б. Язык, онтология и реализм. [Текст] / Л. Б. Макеева; Нац. исслед. ун-т «Высшая школа экономики». -- М.: Изд. дом Высшей школы экономики, 2011. --
	310, [2] с. -- 600 экз. -- ISBN 978-5-7598-0802-2 (в пер.).
	
	\bibitem{gruber} Gruber, T. R. A Translation Approach to Portable Ontologies. -- Knowledge Acquisition -- 1993. -- № 5(2). -- p. 199–220.
	
	\bibitem{ontology_guide} Noy, N.F. Ontology Development 101: A Guide to Creating Your First Ontology / N.F. Noy, D.L. McGuinness -- Stanford Knowledge Systems Laboratory Technical Report KSL-01-05 and Stanford Medical Informatics Technical Report. -- 2001. -- SMI-2001-0880, p. 1-25.
	
	\bibitem{solution_task} Пальчунов, Д.Е. Решение задачи поиска информации на основе онтологий [Текст] / Бизнес-информатика -- 2008. -- № 1. -- С. 3-13.
	
	\bibitem{lingvocabulary} Лингвистический энциклопедический словарь [Текст] / Под ред. В. Н. Ярцевой. -- М.: Советская энциклопедия, 1990. -- 685 с.
	
	\bibitem{auto_processing_nl} Большакова, Е.И Автоматическая обработка текстов на естественном языке и анализ данных : учеб. пособие [Текст] / Е.И. Большакова, К.В. Воронцов, Н.Э. Ефремова, Э.С. Клышинский, Н.В. Лукашевич, А.С. Сапин -- М.: Изд-во НИУ ВШЭ, 2017. -- 269 с. -- ISBN 978-5-9909752-1-7.
	
	\bibitem{natural_lang_processing} Elizabeth, D. Natural Language Processing. -- Center for Natural Language Processing. -- 2001. 
	
	\bibitem{evaluation_preprocessing} Srividhya, V. Evaluating Preprocessing Techniques in Text Categorization / V. Srividhya, R. Anitha -- International Journal of Computer Science and Application Issue -- 2010 -- p. 49-51. -- ISSN 0974-0767.
	
	\bibitem{auto_processing_nl_cl} Большакова, Е.И. Автоматическая обработка текстов на естественном языке и компьютерная лингвистика : учеб. пособие [Текст] / Е.И. Большакова, Э.С. Клышинский, Д.В. Ландэ, А.А. Носков, О.В. Пескова, Е.В. Ягунова —- М.: МИЭМ, 2011. —- 272 с. -- ISBN 978–5–94506–294–8.
	
	\bibitem{review} Пархоменко, П. А. Обзор и экспериментальное сравнение методов кластеризации текстов [Текст]/ П. А. Пархоменко, А. А. Григорьев, Н. А. Астраханцев -- Труды ИСП РАН, 2017 -- том 29, выпуск 2 -- С. 161–200.
	
	\bibitem{an_information} Akiko Aizawa An information-theoretic perspective of tf–idf measures -- Information Processing and Management. -- 2003 -- p. 45-65.
	
	\bibitem{using_tf_idf} Ramos, J. Using TF-IDF to Determine Word Relevance in Document Queries [Электронный ресурс]. -- Режим доступа: https://www.researchgate.net/publication/228818851\_Using\_TF\_IDF\_to\_determi
	ne\_word\_relevance\_in\_document\_queries (Дата обращения: 02.12.2021).
	
	\bibitem{system_development} Зиберт, А.О. Разработка системы определения наличия заимствований в работах студентов высших учебных заведений. Алгоритмы поиска нечетких дубликатов [Текст] / А.О. Зиберт, В.И. Хрусталев --  
	Universum: Технические науки : электрон. научн. журн. -- 2014. -- № 3 (4).
	
	\bibitem{search_methods} Квашина, Ю.А. Методы поиска дубликатов скомпонованных текстов научной стилистики [Текст] /  Ю.А. Квашина. -- Технологический аудит. -- 2013. -- № 3/1(11) -- С. 16-20.
	
	\bibitem{compare} Зеленков, Ю.Г. Сравнительный анализ методов определения нечетких дубликатов для WEB-документов [Текст] / Ю.Г. Зеленков, И.В. Сегалович -- Труды 9-ой Всероссийской научной конференции «Электронные библиотеки: перспективные методы и технологии, электронные коллекции» RCDL'2007: Сб. работ участников конкурса. -- Т. 1. -- Переславль Залесский: <<Университет города Переславля>>, 2007. -- С. 166—174.
	
	\bibitem{shingle_method} Цимбалов, А.В. Метод шинглов [Текст
	] / А.В. Цимбалов, О.В. Золотарев -- Вестник. -- 2016. -- Серия «Сложные системы: модели, анализ и управление». Выпуск 4. -- С. 72-79.
	
	\bibitem{cos} Преображенский, Ю.П. О методах создания рекомендательных систем [Текст] / Ю.П. Преображенский, В. М. Коновалов -- Вестник Воронежского института высоких технологий. -- 2019. -- № 4(31). -- С.75-79.
	
	\bibitem{network} Бабкин, Э.А. Принципы и алгоритмы искусственного интеллекта:	Монография / Э.А. Бабкин, О.Р. Козырев, И.В. Куркина. – Н. Новгород: Нижегород. гос. техн. ун-т. 2006. 132 с. 
	
	\bibitem{syntree} Теньер Л. Основы структурного синтаксиса: Пер. с фр. – Прогресс, 1988.
	
	\bibitem{examples} Национальный корпус русского языка. 2003—2022 [Электронный ресурс]. -- Режим доступа: ruscorpora.ru (Дата обращения: 04.04.2022).
	
	\bibitem{exampleTree} Национальный корпус русского языка. Синтаксический корпус. 2003—2022 [Электронный ресурс]. -- Режим доступа: https://ruscorpora.ru/new/search-syntax.html (Дата обращения: 04.04.2022).
	
	\bibitem{softmax}Маршаков Д. В. СРАВНЕНИЕ РЕЗУЛЬТАТОВ НЕЙРОСЕТЕВОЙ КЛАССИФИКАЦИИ С ПРИМЕНЕНИЕМ SOFTMAX И ФУНКЦИИ РАССТОЯНИЯ //Математические методы в технологиях и технике. – 2021. – №. 8. – С. 75-78.
	
	\bibitem{python} Документация по Python 3 [Электронный ресурс]. Режим доступа: https://docs.python.org/3/ (Дата обращения 01.02.2022)
	
	\bibitem{pycharm} Документация по PyCharm [Электронный ресурс]. Режим доступа: https://www.jetbrains.com/pycharm/guide/tips/quick-docs/ (Дата обращения 01.02.2022)
	
	\bibitem{discord} Документация по Discord [Электронный ресурс]. Режим доступа: https://support.discord.com/hc/ru (Дата обращения 05.02.2022)
	
	\bibitem{discordS} Аналитика трафика и доля рынка Discord [Электронный ресурс]. Режим доступа: https://www.similarweb.com/ru/website/discord.com/ (Дата обращения 20.05.2022)
	
	\bibitem{javascript} Документация по Javascript [Электронный ресурс]. Режим доступа: https://javascript.ru/manual (Дата обращения 02.02.2022)
	
	\bibitem{pyqt5} Документация по PyQt5 [Электронный ресурс]. Режим доступа: https://www.riverbankcomputing.com/static/Docs/PyQt5/ (Дата обращения 04.04.2022)
	
	\bibitem{qt} Документация по Qt [Электронный ресурс]. Режим доступа:  https://doc.qt.io/ (Дата обращения 04.04.2022)
	
	\bibitem{qtdesigner} Документация по Qt Designer [Электронный ресурс]. Режим доступа: https://doc.qt.io/qt-5/qtdesigner-manual.html (Дата обращения 04.04.2022)
	
	\bibitem{networkx} Документация по NetworkX [Электронный ресурс]. Режим доступа: https://networkx.org/documentation/stable/tutorial.html (Дата обращения 28.03.2022)
	
	\bibitem{speech_rec} Документация по SpeechRecognition [Электронный ресурс]. Режим доступа: https://pypi.org/project/SpeechRecognition/ (Дата обращения 15.04.2022)
	
	\bibitem{natasha} Документация по Natasha [Электронный ресурс]. Режим доступа: https://github.com/natasha/natasha (Дата обращения 07.04.2022)
		
	\bibitem{scipy} Документация по SciPy [Электронный ресурс]. Режим доступа: https://docs.scipy.org/doc/scipy/ (Дата обращения 10.05.2022)
	
	\bibitem{time} Документация по time [Электронный ресурс]. Режим доступа: 	https://docs.python.org/3/library/time.html (Дата обращения 20.05.2022)
	
	
	
	
\end{thebibliography}
\endgroup

\pagebreak